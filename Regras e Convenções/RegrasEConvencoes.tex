\documentclass[12pt,a4paper]{article}
\usepackage[utf8]{inputenc}
\usepackage[T1]{fontenc}
\usepackage{hyperref}
\usepackage{listings}
\usepackage{xcolor}

% Configuração para destacar código
\lstset{
    basicstyle=\ttfamily\footnotesize,
    keywordstyle=\color{blue}\bfseries,
    stringstyle=\color{red},
    commentstyle=\color{green!70!black},
    frame=single,
    numbers=left,
    numberstyle=\tiny,
    breaklines=true,
    captionpos=b,
    escapeinside={\%*}{*)}
}

\title{Regras e Convenções de Escrita de Código}
\author{for project mouse}

\begin{document}

\maketitle

\section*{Introdução}
Este documento descreve regras e convençes para escrita de código em linguagens \textbf{C++}, \textbf{CSHARP} e \textbf{GDScript}. A adoção de padrões consistentes melhora a legibilidade e manutenção do código. Serão abordados estilos como \textbf{PEP8}, \textbf{PascalCase} e \textbf{snake\_case}.

\section{Regras Gerais}
\begin{itemize}
    \item Utilize indentação consistente (recomenda-se 4 espaços por nvel (TAB padrão do vscode).
    \item Evite linhas muito longas (entre 80-120 caracteres).
    \item Insira comentários para explicar lógica complexa ou decisões importantes.
    \item Nomeie variáveis e funções de forma descritiva e consistente.
    \item Separe código em módulos ou classes para organizar funcionalidades.
    \item Escreva TUDO em português exceto convenções decididas ao longo do projeto.
    \item Evite sempre o uso de váriaveis globias, principalmente em partes maiores e mais complexas da aplicação
\end{itemize}

\section{C++}
C++ utilizaremos convenções variadas. Recomenda-se seguir estas práticas gerais até segunda ordem:
\subsection*{Tipo}
\begin{itemize}
    \item Classes: \textbf{PascalCase} (e.g., \lstinline|class MinhaClasse|).
    \item Funções: \textbf{camel\_Case like} (e.g., \lstinline|void minha_Funcao()|).
    \item Variáveis: \textbf{snake\_case} (e.g., \lstinline|int minha_var|).
    \item Constantes: \textbf{UPPER\_SNAKE\_CASE} (e.g., \lstinline|const int MAX_VALUE = 100|).
\end{itemize}

\subsection*{Exemplo de Código}
\begin{lstlisting}[language=C++]
#include <iostream>

class MinhaClasse {
public:
    void print_Mensagem() {
        std::cout << "Hello, World!" << std::endl;
    }
};

int main() {
    MinhaClasse minha_classe;
    minha_classe.print_Mensagem();
    return 0;
}
\end{lstlisting}

\section{CSHARP}
CSHARP seguiremos convenções muito próximas das práticas .NET, priorizando \textbf{PascalCase} para muitos elementos.
\subsection*{Tipo}
\begin{itemize}
    \item Classes e métodos: \textbf{PascalCase} (e.g., \lstinline|public class MinhaClasse|, \lstinline|void MeuMetodo()|).
    \item Variáveis locais e parâmetros: \textbf{camelCase} (e.g., \lstinline|int minhaVar|).
    \item Constantes: \textbf{UPPER: PASCALCase} (e.g., \lstinline|const int MAXValue = 100|).
    \item Campos privados: Prefixados com \_ (e.g., \lstinline|private int _meuPrivado|).
\end{itemize}

\subsection*{Exemplo de Código}
\begin{lstlisting}[language=C++]
using System;

namespace MyNamespace {
    public class MinhaClasse {
        private int _meuPrivado;

        public void MeuMetodo() {
            Console.WriteLine("Hello, World!");
        }
    }

    class Programa {
        static void Main(string[] args) {
            MinhaClasse minhaClasse = new MinhaClasse();
            MinhaClasse.MeuMetodo();
        }
    }
}
\end{lstlisting}

\section{GDScript}
GDScript, utilizado no Godot Engine, segue o estilo \textbf{snake\_case} em geral, com algumas variações específicas para scripts.
\subsection*{Tipo}
\begin{itemize}
    \item Funções e variáveis: \textbf{snake\_case} (e.g., \lstinline|func minha_funcao():|, \lstinline|var minha_var|).
    \item Classes: \textbf{PascalCase} (e.g., \lstinline|class_name MinhaClasse|).
    \item Constantes: \textbf{UPPER\_SNAKE\_CASE} (e.g., \lstinline|const MAX_SPEED = 100|).
\end{itemize}

\subsection*{Exemplo de Código}
\begin{lstlisting}[language=Python]
extends Node

class_name MinhaClasse

    const MAX_SPEED = 100
    
    var minha_var = 10
    
    func _comecar():
        print("Hello, World!")
    
    func minha_funcao():
        return MAX_SPEED * minha_var
\end{lstlisting}

\section{Comentários}
Durante todo o código comente o necessário para que qualquer um seja capaz de entender o que foi feito. Siga o seguinte padrão para comentar: 
\begin{itemize}
    \item Funções: Comente acima da declaração da função o que e como ela faz. Explique sua entrada e seu retorno abaixo da declaração da função.
    \item Classes: Comente acima da declaração da classe para que ela serve. Comente abaixo da declaração da classe todos os seus campos.
    \item Constantes: Devem ter nomes descritivos o suficiente para que não seja necessário comentá-los, exceto caso esteja utilizando uma constante como flag, portanto deve-se explicar como e porque utiliza essa flag.
    \item Loops: Deve-se comentar ao lado da declaração do loop o que ele faz e como ele faz
    \item Sinalização para os outros devs: Deve-se sinalizar caso alguma parte do código não deva ser alterada e porque. Deve-se sinalizar caso alguma parte do código deva ser alterada, como e porque.
\end{itemize}

\section*{Conclusão}
Seguir convenções de escrita facilita a colaboração e a leitura do código. Consistência é essencial para projetos profissionais e pessoais. Perceba que seguindo esse padrão de convenções durante o projeto, saberemos em qual linguagem e com o que estamos trabalhando mais facilmente.

\end{document}
